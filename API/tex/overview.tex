\begin{section}{Overview}
  
We introduce some of the core concepts and basic terminology for the \cPanel API.
    
  \begin{subsection}{Core Concepts}
    \begin{itemize}
    \item \define{Observer Pattern}
      \begin{itemize}
      \item The design pattern that the \cPanel API implements.
      \item Consists of two categories, {\em subjects} and {\em observers}.
      \item When the subject changes state, the registered observers are notified automatically.
      \item In \cPanel, the subjects are {\em hookable events} and the observers are {\em hooks}.
      \end{itemize}
    \item \define{Hookable Events}
      \begin{itemize}
      \item Specific points in \cPanel and \WHM where customized code can be executed.
      \item The \cPanel API provides an interface for writing code that works with \cPanel events.
      \item Examples include creation/removal of an account or the beginning/end of an update or backup.
      \item May permit the escalation of privileges.
      \item May force execution of hooks to run as a specific user.
      \item May permit a hook to block execution of further \cPanel code upon failure.
      \end{itemize}
    \item \define{Hooks}
      \begin{itemize}
      \item Customized code registered with a hookable event.
      \item Specified to run as a specific user on the system, defers to the event for privilege escalation.
      \item May be written as a script (perl/php) or a perl module.
      \end{itemize}
    \item \define{API 1}
      \begin{itemize}
      \item Targets \cPanel versions 11.30 and lower.
      \item Primarily useful for writing skins or themes.
      \end{itemize}
    \item \define{API 2}
      \begin{itemize}
      \item Recommend API to use for \cPanel version 11.30 and later.
      \end{itemize}
    \end{itemize}
  \end{subsection}  
\end{section}
